\documentclass{article}
\author{Marco Meyer}
\title{CS 102 \LaTeX  \"Ubung 5}
\date{25. Oktober 2015}
\begin{document}

\maketitle
\section{Dies ist der erste Abschnitt}
Hier k\"onnte auch anderer Text stehen, zum Beispiel eine Einleitung.

\section{Tabelle}
Meine unwichtigsten Daten finden Sie in folgender Tabelle 1.\\

\begin{tabular}[c]{c|c|c|c}

 & Punkte erhalten & Punkte m\"oglich & \%\\
\hline
Aufgabe 1 & 2 & 4 &0.5 \\
Aufgabe 2 & 3 & 3 &1\\
Aufgabe 3 & 3 & 3 &1 \\

\end{tabular} \\
\begin{center}
Tabelle 1: Diese Tabelle enthält jetzt jedoch trotzdem dieselben Daten.
\end{center} 

\section{Formeln}
\subsection{Pythagoras}
Der Satz des Pythagoras errechnet sich wie folgt: $ a^{2}+b^{2}=c^{2} $. Daraus k\"onnen wir die L\"ange der Hypothenuse wie folgt bestimmen: $ c=\sqrt{a^{2}+b^{2}} $.
\subsection{Summen}
Wir k\"onnen auch die Formel f\"ur eine Summe angeben:\\ 
\begin{equation}
	s=\sum_{i=1}^n i=\frac{n*(n+1)}{2}
	\label{eq:formel}
\end{equation} 
\textbf{Blubblub Sven Freimann war hier!!!}
\end{document} 
